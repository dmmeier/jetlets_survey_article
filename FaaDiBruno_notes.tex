\documentclass[12pt]{amsart}
\usepackage{amsmath,amssymb}
\usepackage{geometry} % see geometry.pdf on how to lay out the page. There's lots.
\geometry{a4paper} % or letter or a5paper or ... etc
% \geometry{landscape} % rotated page geometry

%  POSSIBLY USEFULE PACKAGES
%\usepackage{graphicx}
%\usepackage{tensor}
%\usepackage{todonotes}

%  NEW COMMANDS
\newcommand{\pder}[2]{\ensuremath{\frac{ \partial #1}{\partial #2}}}
\newcommand{\ppder}[3]{\ensuremath{\frac{\partial^2 #1}{\partial
      #2 \partial #3} } }
\newcommand{\R}{\ensuremath{\mathbb{R}}}

%  NEW THEOREM ENVIRONMENTS
\newtheorem{thm}{Theorem}
\newtheorem{prop}[thm]{Proposition}
\newtheorem{cor}[thm]{Corollary}
\newtheorem{lem}[thm]{Lemma}
\newtheorem{defn}[thm]{Definition}


%  MATH OPERATORS
\DeclareMathOperator{\Diff}{Diff}
\DeclareMathOperator{\GL}{GL}
\DeclareMathOperator{\SO}{SO}
\DeclareMathOperator{\ad}{ad}
\DeclareMathOperator{\Ad}{Ad}

%  TITLE, AUTHOR, DATE
\title[Fa\`a di Bruno, without losing your marbles]
{How to stare at the $n$-dimensional Fa\`a di Bruno formula
without losing your marbles}
\author{Henry O. Jacobs}
\date{\today}


\begin{document}

\maketitle

Let $f,g: \R \to \R$.
The formula for the $k$th derivative of $f\circ g$ is
\begin{align}
  \begin{split}
  &\frac{d^m}{dx^m} (f \circ g)(x) = \\
  &\quad
  \sum_{k=1}^{n}
  \left(
    \sum_{
      \substack{
        b_1 + \cdots + b_m = k \\
        b_1 + 2 b_2 + \cdots + m b_m = m
      }
    }
      \frac{m!}{b_1! \cdots b_m !} g^{(k)}(f(x)) 
      \left( \frac{f^{(1)}(x)}{1!} \right)^{b_1}
      \cdots
      \left( \frac{f^{(m)}(x)}{m!} \right)^{b_n}
  \right)
  \end{split}\label{eq:Faa1}
\end{align}
A typical reaction to a first encounter with \eqref{eq:Faa1}
is described in the first page of \cite{Flanders2001} wherein
the author was asked to provide a proof:

\begin{quotation}
  This excercise is hardly routine calculus!  [...]
  All those factorials in the denominators, raised to powers yet!
  My four years of Chicago high school mathematics: Algebra, Advanced Algebra, [...], calculus hardly prepared me for Fa\`a's formula.
\end{quotation}
Fortunately, \cite{Flanders2001} manages to make sense of the formula; by
justifying the restrictions on the $b$'s and all those factorials.
In fact, over the past two centuries \eqref{eq:Faa1} has been
viewed from a variety of perspectives; such as Bell polynomials, set partitions, determinant formulas, and so on \cite{Johnson2002}.

However, there is very little written on the multi-dimensional 
generalization of \eqref{eq:Faa1}.
To provide perspective on the issue, 
we can state the multi-dimensional
Faa-di-Bruno formula at orders $1,2$, and $3$.
Let $\partial_{i}$ denote the partial differentiation operator
along the $i$th coordinate direction,
and let $\partial_{ij} = (\partial_i \circ \partial_j), \partial_{ijk} = (\partial_i \circ \partial_j \circ \partial_k)$.
If $g = (g^1,\dots,g^d) : \R^c \to \R^d$ and $f:\R^d \to \R$, then
\begin{align*}
  \partial_i (f \circ g)(x) &= \sum_{j=1}^{d}\partial_jf(g(x)) 
  (\partial_i g^j(x)) \\
  \partial_{ij}(f \circ g)(x) &= \sum_{k,\ell = 1}^{d} \partial_{k\ell}f(g(x))
  (\partial_i g^k(x) ) (\partial_j g^{\ell}(x))
  + \sum_{k=1}^{d}\partial_k f(g(x)) \partial_{ij} g^k(x)
\end{align*}
and finally,
\begin{align*}
  &\partial_{ijk} (f \circ g)(x) = 
  \sum_{\ell,m,n=1}^{d} \partial_{\ell m n} f(g(x)) 
  (\partial_i g^\ell(x))  (\partial_j g^m(x))
  (\partial_k g^n(x)) \\
  &\quad + \sum_{mn=1}^{d} \partial_{mn}f(g(x))
  ( \partial_j g^m(x) )
  (\partial_{ik} g^n(x)) \\
  &\quad + \sum_{mn=1}^{d} \partial_{mn}f(g(x))
  ( \partial_k g^m(x) )
  (\partial_{ij} g^n(x)) \\
  &\quad + \sum_{mn=1}^{d} \partial_{mn}f(g(x))
  ( \partial_i g^m(x) )
  (\partial_{jk} g^n(x)) \\
  &\quad + \sum_{\ell=1}^{d} \partial_\ell f(g(x)) \partial_{ijk}g^\ell(x).
\end{align*}
where $i,j,k = 1,\dots,c$.

It is natural to ask for a generic formula for
$\partial_{i_1 \cdots i_n} (f \circ g)(x)$.
One must consider set partitions
along each dimension,
and account for equivalences such as
``$\partial_{ij} \equiv \partial_{ji}$''.
Simply printing the formula in multi-index
notation is a formidable task
and we are not able to present the multi-indexed formula here.
Instead, we refer the reader to the statement and proof in
 \cite{ConstantineSavits1996},
where one can see the difficulties of the multi-indexed Fa\`a di-Bruno.
The statement and proof of \cite{ConstantineSavits1996}
consists of three full pages, covered almost entirely in 
combinatorial coefficients.
This is in spite of good notation and an efficient proof!

In this paper, we will introduce a version of the multi-dimensional Fa\`a di-Bruno
formula, in which all of the coefficients are equal to $1$ and the proof is
 four lines long.
The main idea is to index partial derivatives by multi-sets
instead of multi-indices.


\section{Unordered multi-indices and bags of marbles}
Rather than using multi-indices, perhaps we could place (hide?) the
combinatoric considerations in the indexing convention itself.
This entails summing over a smaller set of slightly more sophisticated indices.

Let $S_n$ denote the permutation group on $n$ elements.
  Clearly, $S_n$ acts on $\{ 1,2,\dots, d\}^n$ by sending
  \begin{align*}
    (i_1,\dots,i_n) \stackrel{\sigma \in S_n}{\mapsto} ( i_{\sigma_1} ,\dots,i_{\sigma_n}).
  \end{align*}
  An unordered multi-index of degree $n$ is an element of the quotient $\mathbb{I}_d^n := \{1,\dots,d\}^n / S_n$.
  Alternatively, one can define an unordered multi-index as a multiset, or ``bag'', whose underlying set is $\{1,\dots,d\}$.
  Heuristically, an unordered multi-index is nothing but a bag of
  $n$ marbles, which come in colors $1$ through $d$  \cite{Blizard1989}.

Given $(a_1,\dots,a_n) \in \{ 1,\dots,d\}^n$, we let $[a_1 ,\dots, a_n] \in \mathbb{I}_d^n$ denote its $S_n$-orbit.
Given two unordered multi-indices, $\alpha \in \mathbb{I}^n$ and $\beta \in \mathbb{I}_d^m$, we let $[\alpha,\beta]$ denote the corresponding element of $\mathbb{I}_d^{n +m}$,
obtained by taking two representatives $\hat{\alpha} \in \alpha$ and $\hat{\beta} \in \beta$ and then setting
\begin{align*}
  [\alpha,\beta] := \{ \sigma \cdot (\hat{\alpha},\hat{\beta}) \mid \sigma \in S_{n+m}\} \in \mathbb{I}_d^{n+m}.
\end{align*}
Hueristically, $[\alpha,\beta]$ is the bag of $n+m$ marbles obtained
by combing the bags $\alpha$ and $\beta$.
The degree of an unordered multi-index $\alpha \in \mathbb{I}_d^n$ is $n$ and denoted $|\alpha|$.
Note that an unordered multi-index of degree $1$ is an element of $\{1,\dots,d\}$, and that the only unordered multi-index of degree $0$ is the empty set.

  Given an $\alpha \in \mathbb{I}^n$, we let $\partial_{\alpha}$ denote
the partial differential operator $\partial_{a_1 \cdots a_n}$ obtained from an arbitrary element $(a_1,\dots,a_n) \in \alpha$.  Note that the representative element of $\alpha$ is immaterial due to the equivalence of mixed partials.
Given this convention we observe $\partial_{\alpha} \partial_{\beta} = \partial_{\beta} \partial_{\alpha} = \partial_{[\alpha ,\beta]}$ for any two unordered multi-indices $\alpha \in \mathbb{I}_d^n$ and $\beta \in \mathbb{I}_d^m$.
This indexing convention is slightly more complicated than the standard
one, but it makes the formulas regarding partial derivative operators simpler by equating $\partial_{xy} = \partial_{yx}, \partial_{xxy}=\partial_{xyx}=\partial_{yxx} = \partial_x \partial_{xy} = \partial_{xy} \partial_x$, etc.

\begin{thm}
Let $f:\R^c \to \R$ and $g:\R^d \to \R^c$.  Then
\begin{align*}
  \partial_\alpha( f \circ g)(x) = \sum_{n=1}^{|\alpha|} \left(
  \sum_{ 
    \substack{
      \beta = [b_1,\dots,b_n] \in \mathbb{I}^n_c \\
      [\gamma_1 ,\dots, \gamma_n ] = \alpha 
      }
    }
    \partial_\beta f(g(x)) \partial_{\gamma_1} g^{b_1}(x) \cdots \partial_{\gamma_n} g^{b_n}(x)
  \right),
\end{align*}
  for any unorderd multi-index $\alpha \in \mathbb{I}^{m}_d$.
\end{thm}
\begin{proof}
  We prove it inductively.  It holds trivially at order $0$,
  and by inspection at order $1$.
  Assume it holds for $\alpha \in \mathbb{I}^n_d$
  and let $a_0 \in \{ 1,\dots,d\} \equiv \mathbb{I}_d^1$.
  By the product formula and chain rule, we find
  \begin{align*}
    &\partial_{a_0}\partial_\alpha( f \circ g)(x) = 
    \partial_{[a_0,\alpha]}( f \circ g)(x) \\
    &\quad =\sum_{n=1}^{|\alpha|}
    \sum_{ 
      \substack{
        \beta = [b_1,\dots,b_n] \in \mathbb{I}_c^n \\
        [\gamma_1,\dots,\gamma_n] = \alpha 
        }
      }
      \left( \sum_{b_0=1}^d
        \partial_{[b_0,\beta]} f(g(x))
        \partial_{a_0} g^{b_0}(x)
        \partial_{\gamma_1} g^{b_1}(x) \cdots
        \partial_{\gamma_n} g^{b_n}(x)
      \right)\\
      &\qquad + \partial_{\beta} f(g(x))
      \partial_{[a_0,\gamma_1]} g^{b_1}(x) \cdots
      \partial_{\gamma_n} g^{b_n}(x) \\
      &\qquad + \partial_{\beta} f(g(x))
      \partial_{\gamma_1} g^{b_1}(x) 
      \partial_{[a_0,\gamma_2]} g^{\beta_2}(x)  \cdots
      \partial_{\gamma_n} g^{b_n}(x) \\
      &\qquad \vdots \\
      &\qquad + \partial_{\beta} f(g(x))
      \partial_{\gamma_1} g^{b_1}(x) \cdots
      \partial_{[a_0,\gamma_n]} g^{b_n}(x) . 
  \end{align*}
We now collect terms involving $\beta$'s of a single order.
To do so, collect all coefficients of of $\partial_{[b_0,\dots,b_n]} f(g(x))$.
We observe that this coefficient is
\begin{align*}
&   \left( \sum_{ [\gamma_1,\dots,\gamma_n] = \alpha}
   \partial_{a_0}g^{b_0}(x) \partial_{\gamma_1} g^{b_1}(x) \cdots \partial_{\gamma_n}g^{b_n}(x) \right) \\
& +
 \left( \sum_{k=1}^{n} \sum_{ [\gamma_0, \dots,\gamma_n] = \alpha}
  \partial_{\gamma_0}g^{b_0} \cdots \partial_{[a_0,\gamma_k]} g^{\beta_k} \cdots \partial_{\gamma_n} g^{b_n}\right),
\end{align*}
which one could write more succinctly as
\begin{align*}
  \sum_{[\gamma_0,\dots,\gamma_n] = [a_0,\alpha]}
  \partial_{\gamma_0}g^{b_0}(x)
  \cdots
  \partial_{\gamma_n}g^{b_n}(x).
\end{align*}
We substitute the coefficent of
$\partial_{\beta} f(g(x))$ into our original formula to arrive at
\begin{align*}
  \partial_{[a_0,\alpha]}( f \circ g)(x) &= \sum_{n=0}^{|\alpha |} \sum_{
    \substack{
      \beta = [b_0,\dots,b_n] \in \mathbb{I}_c^{n+1}\\
      [\gamma_0,\dots,\gamma_n] = [a_0,\alpha]      
      }
    }\partial_{\beta}f(g(x)) \partial_{\gamma_0}g^{b_0}(x) \cdots \partial_{\gamma_n}g^{b_n}(x) \\
  &=\sum_{n=1}^{|[a_0,\alpha] |} \sum_{
    \substack{
      \beta = [b_1,\dots,b_n] \in \mathbb{I}_c^{n}\\
      [\gamma_1,\dots,\gamma_n] = [a_0,\alpha]      
      }
    }\partial_{\beta}f(g(x)) \partial_{\gamma_1}g^{b_1}(x) \cdots \partial_{\gamma_n}g^{b_n}(x).
\end{align*}
\end{proof}


\section{Conclusion}
While the $n$-dimensional Fa\`a di Bruno formula may be difficult
to tackle when using multi-indices, it appears
relatively simple when using unordered multi-indices.
This is not to say that unordered multi-indices are neccessarily
superior.
The summation constraint
``$[\gamma_1,\dots,\gamma_n] = \alpha$''
can be difficult to enforce in an algorithm.
It is quite conceivable that one would prefer to 
enumerate all the multi-indices and then divide by the 
number of repeated terms.
However, the combinatorial coefficients can be unweildly,
so it is useful to have an alternative which disposes of them.
This is particularly useful outside of computing,
when one must invoke the formula in a proof.

\bibliographystyle{amsalpha}
\bibliography{/Users/hoj201/Dropbox/hoj_2014.bib}
\end{document}
