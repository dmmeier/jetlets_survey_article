\documentclass[12pt]{amsart}
\usepackage{amsmath,amssymb}
\usepackage{geometry} % see geometry.pdf on how to lay out the page. There's lots.
\geometry{a4paper} % or letter or a5paper or ... etc
% \geometry{landscape} % rotated page geometry

%  POSSIBLY USEFULE PACKAGES
%\usepackage{graphicx}
%\usepackage{tensor}
%\usepackage{todonotes}

%  NEW COMMANDS
\newcommand{\pder}[2]{\ensuremath{\frac{ \partial #1}{\partial #2}}}
\newcommand{\ppder}[3]{\ensuremath{\frac{\partial^2 #1}{\partial
      #2 \partial #3} } }
\newcommand{\R}{\ensuremath{\mathbb{R}}}

%  NEW THEOREM ENVIRONMENTS
\newtheorem{thm}{Theorem}[section]
\newtheorem{prop}[thm]{Proposition}
\newtheorem{cor}[thm]{Corollary}
\newtheorem{lem}[thm]{Lemma}
\newtheorem{defn}[thm]{Definition}


%  MATH OPERATORS
\DeclareMathOperator{\Diff}{Diff}
\DeclareMathOperator{\GL}{GL}
\DeclareMathOperator{\SO}{SO}
\DeclareMathOperator{\ad}{ad}
\DeclareMathOperator{\Ad}{Ad}

%  TITLE, AUTHOR, DATE
\title{The Faa Di Bruno Formula in multi-index notation}
\author{Henry O. Jacobs}
\date{\today}


\begin{document}

\maketitle

Those irritating combinatorial coefficients in the Faa-di-Bruno formula
come from the fact that the standard multi-indices don't quite model
partial derivative operators, and so one must be careful not to double 
count things.
Rather than using the standard multi-indices as elemenst of $\mathbb{N}^n$ or
$\{1,\dots,d\}^n$,
it seems easier to put all of the combinatorics into the indices themselves (i.e. summing over a smaller set of slightly more sophisticated indices to avoid double counting).
This desire inspired the following multi-index conventions on $\R^d$.  Let $S_n$
  denote the permutation group on $n$ elements.
  Clearly $S_n$ acts on $\{ 1,\dots, d\}^n$ by sending
  \begin{align*}
    (i_1,\dots,i_n) \stackrel{\sigma \in S_n}{\mapsto} ( i_{\sigma_1} ,\dots,i_{\sigma_n}).
  \end{align*}

  A multi-index of degree $n$ is an element $\alpha$ of the quotient $\mathbb{I}^n := \{1,\dots,d\}^n / S_n$.
Given $\alpha \in \mathbb{I}^n$ and $\beta \in \mathbb{I}^m$ we let $[\alpha,\beta]$ denote the corresponding element of $\mathbb{I}^{n +m}$
obtained by taking two representatives $\hat{\alpha} \in \alpha$ and $\hat{\beta} \in \beta$ and then setting $[\alpha,\beta] = \pi_{S_{n+m}}(\hat{\alpha},\hat{\beta})$.
The degree of a multi-index $\alpha \in \mathbb{I}^n$ is $n$ and denoted $|\alpha|$.  Note that a multi-index of degree $1$ is merely an element of $\{1,\dots,d\}$, and that there is no such thing as a multi-index of degree $0$.

This indexing convention is slightly more complicated than the standard
one, but it makes the formulas regarding partial derivative operators much more simple by equating $\partial_{xxy}=\partial_{xyx}=\partial_{yxx} = \partial_x \partial_{xy} = \partial_{xy} \partial_x$, etc.  In particiular $\partial_{\alpha} \partial_{\beta} = \partial_{\beta} \partial_{\alpha} = \partial_{[\alpha,\beta]}$ for any two multi-indices $\alpha \in \mathbb{I}^n$ and $\beta \in \mathbb{I}^m$.


\begin{thm}
The Faa-di-Bruno formula with respect to this indexing convention is
\begin{align*}
  \partial_\alpha( \varphi \circ \psi)(x) = \sum_{n=1}^{|\alpha|} \left(
  \sum_{ 
    \substack{
      \beta = [\beta_1,\dots,\beta_n] \in \mathbb{I}^n \\
      [\gamma_1,\dots,\gamma_n] = \alpha 
      }
    }
    \partial_\beta \varphi(\psi(x)) \partial_{\gamma_1} \psi^{\beta_1}(x) \cdots \partial_{\gamma_n} \psi^{\beta_n}(x)
  \right) 
\end{align*}
\end{thm}
\begin{proof}
  We prove it inductively.  It holds trivially at order $0$,
  and by inspection at order $1$.
  Assume it holds for $\alpha \in \mathbb{I}^n$
  and let $\alpha_0 \in \{ 1,\dots,d\}$.
  By the product formula and chain rule we find
  \begin{align*}
    &\partial_{\alpha_0}\partial_\alpha( \varphi \circ \psi)(x) = 
    \partial_{[\alpha_0,\alpha]}( \varphi \circ \psi)(x) \\
    &\quad =\sum_{n=1}^{|\alpha|}
    \sum_{ 
      \substack{
        \beta = [\beta_1,\dots,\beta_n] \in \mathbb{I}^n \\
        [\gamma_1,\dots,\gamma_n] = \alpha 
        }
      }
      \left( \sum_{\beta_0=1}^d
        \partial_{[\beta_0,\beta]} \varphi(\psi(x))
        \partial_{\alpha_0} \psi^{\beta_0}(x)
        \partial_{\gamma_1} \psi^{\beta_1}(x) \cdots
        \partial_{\gamma_n} \psi^{\beta_n}(x)
      \right)\\
      &\qquad + \partial_{\beta} \varphi(\psi(x))
      \partial_{[\alpha_0,\gamma_1]} \psi^{\beta_1}(x) \cdots
      \partial_{\gamma_n} \psi^{\beta_n}(x) \\
      &\qquad + \partial_{\beta} \varphi(\psi(x))
      \partial_{\gamma_1} \psi^{\beta_1}(x) 
      \partial_{[\alpha_0,\gamma_2]} \psi^{\beta_2}(x)  \cdots
      \partial_{\gamma_n} \psi^{\beta_n}(x) \\
      &\qquad \vdots \\
      &\qquad + \partial_{\beta} \varphi(\psi(x))
      \partial_{\gamma_1} \psi^{\beta_1}(x) \cdots
      \partial_{[\alpha_0,\gamma_n]} \psi^{\beta_n}(x) . 
  \end{align*}
We now collect terms involving $\beta$'s of a single order.
To do that, collect all coefficients of of $\partial_{[\beta_0,\dots,\beta_n]} \varphi(\psi(x))$.
We observe this coefficient is
\begin{align*}
&   \left( \sum_{ [\gamma_1,\dots,\gamma_n] = \alpha}
   \partial_{\alpha_0}\psi^{\beta_0}(x) \partial_{\gamma_1} \psi^{\beta_1}(x) \cdots \partial_{\gamma_n}\psi^{\beta_n}(x) \right) \\
& +
 \left( \sum_{k=1}^{n} \sum_{ (\gamma_0, \dots,\gamma_n) = \alpha}
  \partial_{\gamma_0}\psi^{\beta_0} \cdots \partial_{[\alpha_0,\gamma_k]} \psi^{\beta_k} \cdots \partial_{\gamma_n} \psi^{\beta_n}\right)
\end{align*}
which you could write more succinctly as
\begin{align*}
  \sum_{[\gamma_0,\dots,\gamma_n] = [\alpha_0,\alpha]}
  \partial_{\gamma_0}\psi^{\beta_0}(x)
  \cdots
  \partial_{\gamma_n}\psi^{\beta_n}(x)
\end{align*}
To drive the point home, we now substitute the coefficent of
$\partial_{\beta} \varphi(\psi(x))$, written in this form,
into our original formula to find
\begin{align*}
  \partial_{[\alpha_0,\alpha]}( \varphi \circ \psi)(x) &= \sum_{n=0}^{|\alpha |} \sum_{
    \substack{
      \beta = (\beta_0,\dots,\beta_n) \in \mathbb{I}^{n+1}\\
      [\gamma_0,\dots,\gamma_n] = [\alpha_0,\alpha]      
      }
    }\partial_{\beta}\varphi(\psi(x)) \partial_{\gamma_0}\psi^{\beta_0}(x) \cdots \partial_{\gamma_n}\psi^{\beta_n}(x) \\
  &=\sum_{n=1}^{|[\alpha_0,\alpha] |} \sum_{
    \substack{
      \beta = (\beta_1,\dots,\beta_n) \in \mathbb{I}^{n}\\
      [\gamma_1,\dots,\gamma_n] = [\alpha_0,\alpha]      
      }
    }\partial_{\beta}\varphi(\psi(x)) \partial_{\gamma_0}\psi^{\beta_0}(x) \cdots \partial_{\gamma_n}\psi^{\beta_n}(x).
\end{align*}

\end{proof}

\bibliographystyle{amsalpha}
\bibliography{/Users/hoj201/Dropbox/hoj_2014.bib}
\end{document}
